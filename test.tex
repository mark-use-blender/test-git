
Let \( E \) be the ellipse given by the equation:

\[
\frac{x^2}{a^2} + \frac{y^2}{b^2} = 1
\]

where \( a, b > 0 \). We aim to find the area bounded by the ellipse.

---

### Step-by-Step Solution

#### **1. Understanding the Ellipse**
The equation of the ellipse is in standard form:

\[
\frac{x^2}{a^2} + \frac{y^2}{b^2} = 1
\]

Here, \( a \) and \( b \) are the semi-major and semi-minor axes, respectively. The ellipse extends from \( -a \) to \( a \) along the x-axis and from \( -b \) to \( b \) along the y-axis.

---

#### **2. Deriving the Area**

The area \( A \) of an ellipse can be found using integration or parametrization. We will use integration for this derivation.

##### **a. Solving for \( y \) in terms of \( x \)**

From the equation:

\[
\frac{y^2}{b^2} = 1 - \frac{x^2}{a^2}
\]

Multiply both sides by \( b^2 \):

\[
y^2 = b^2 \left(1 - \frac{x^2}{a^2}\right)
\]

Take the square root:

\[
y = \pm b \sqrt{1 - \frac{x^2}{a^2}}
\]

This gives us the upper and lower halves of the ellipse.

##### **b. Setting up the Integral**

The area \( A \) can be computed by integrating the function representing the upper half of the ellipse from \( x = -a \) to \( x = a \), and then doubling the result (to account for the lower half):

\[
A = 2 \int_{-a}^{a} b \sqrt{1 - \frac{x^2}{a^2}} \, dx
\]

Since the integrand is even (symmetric about the y-axis), we can simplify by integrating from \( 0 \) to \( a \) and multiplying by 2:

\[
A = 4b \int_{0}^{a} \sqrt{1 - \frac{x^2}{a^2}} \, dx
\]

##### **c. Substitution**

Let’s perform the substitution:

\[
x = a \sin \theta \quad \Rightarrow \quad dx = a \cos \theta \, d\theta
\]

When \( x = 0 \), \( \theta = 0 \). When \( x = a \), \( \theta = \frac{\pi}{2} \).

Substitute into the integral:

\[
A = 4b \int_{0}^{\frac{\pi}{2}} \sqrt{1 - \sin^2 \theta} \cdot a \cos \theta \, d\theta
\]

Simplify using \( \sqrt{1 - \sin^2 \theta} = \cos \theta \):

\[
A = 4ab \int_{0}^{\frac{\pi}{2}} \cos^2 \theta \, d\theta
\]

##### **d. Using a Trigonometric Identity**

Use the identity \( \cos^2 \theta = \frac{1 + \cos(2\theta)}{2} \):

\[
A = 4ab \int_{0}^{\frac{\pi}{2}} \frac{1 + \cos(2\theta)}{2} \, d\theta
\]

Simplify:

\[
A = 2ab \int_{0}^{\frac{\pi}{2}} (1 + \cos(2\theta)) \, d\theta
\]

##### **e. Integrating**

Integrate term by term:

\[
\int_{0}^{\frac{\pi}{2}} 1 \, d\theta = \left[ \theta \right]_0^{\frac{\pi}{2}} = \frac{\pi}{2}
\]

\[
\int_{0}^{\frac{\pi}{2}} \cos(2\theta) \, d\theta = \left[ \frac{\sin(2\theta)}{2} \right]_0^{\frac{\pi}{2}} = 0
\]

Thus:

\[
A = 2ab \left( \frac{\pi}{2} + 0 \right) = ab \cdot \pi
\]

---

#### **3. Final Answer**

The area bounded by the ellipse is:

\[
A = \boxed{\pi a b}
\]

---

This result shows that the area of an ellipse is \( \pi a b \), where \( a \) and \( b \) are the lengths of the semi-major and semi-minor axes, respectively.
